\documentclass[midterm]{../package/ecpknproposal}
% TODO: (USER) 填写以下的内容
\thesistilte{课题题目}
\title{\buaaatthesistitle}
\authorname{研究生姓名}
\author{\buaaatauthorname}

\tutor{导师姓名 \quad 职称}
\stuno{XXXX}
\major{专业名称}

\ckeyword{可压缩湍流,能量传输,级联}
\ekeyword{compressible turbulence, energy transfer, cascade}

\date{2022年12月15日}

\begin{document}
\pdfbookmark[0]{\buaaatthesistitle 中期报告}{cover}
\maketitle

\linespread{1.5}
\pagestyle{frontmatter}
\maketoc % TODO: (USER)可以更换为 \makefulltoc \maketocandfigure \maketocandtable

\begin{cabstract}
可压缩湍流的能量传输是流体力学中的重要研究课题,对于湍流理论解释与工程应用均有重大意义。本文简要描述了可压缩湍流的基本性质,主要针对能量传输相关的能量级联现象、速度梯度张量结构特征与偏斜因子特征,就一、二、三维不可压缩与可压缩流体中的已有研究成果进行了总结,梳理了目前已有的技术方法,并且就未来研究方向提出了展望。
\end{cabstract}


\begin{eabstract}
The research on the energy transfer of compressible turbulence is of great importance in the fluid mechanics. It is of great significance for the theoretical explanation of turbulence and engineering applications. This article briefly describes the basic properties of compressible turbulent flow. It summaries the research of one, two and three-dimensional incompressible and compressible fluids, mainly focusing on the energy cascade, the characteristics of velocity gradient tensor and the characteristics of skewness factor. It synthesizes at the same time the existing methods, as well as puts forward the possible directions of future researches.
\end{eabstract}

\newpage

\section{绪论}
\subsection{研究背景及意义}

\subsection{国内外研究现状}

\subsection{研究内容及中期进展}


\section{正文第一章}

\section{正文第二章}

\section{正文第三章}

\section{工作总结及展望}
\subsection{待开展相关工作}
\subsection{后续研究计划}


\newpage

\phantomsection
\addcontentsline{toc}{section}{\bibname}
\small
\bibliography{refs}

\nocite{*}
\end{document}
